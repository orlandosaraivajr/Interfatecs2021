\section{Instruções}

Este caderno contém 11 problemas – identificados por letras de A até K, com páginas numeradas de 3 até \pageref{LastPage}. Verifique se seu caderno está completo.

Informações gerais

\begin{enumerate}
\item Sobre a competição
\begin{enumerate}
\item A competição possui duração de 5 horas (início as 13:00h término as 18:00h);
%\item NÃO é permitido acesso a conteúdo da Internet ou qualquer outro meio eletrônico digital;
%\item É permitido somente acesso a conteúdo impresso (cadernos, apostilas, livros);
\item É permitido a consulta a materiais já publicados anteriormente ao dia da competição
\item Não é permitido a comunicação com o técnico ou qualquer outra pessoa que não seja a equipe para tirar dúvidas sobre a maratona
\item É vedada a comunicação entre as equipes durante a competição, bem como a troca de material de consulta entre elas;
%\item Cada equipe terá acesso a 1 computador dotado do ambiente de submissão de programas (BOCA), dos compiladores, link-editores e IDEs requeridos pelas linguagens de programação permitidas;
\item Cada integrante da equipe poderá utilizar o seu computador/notebook para resolver os problemas, porém, apenas um competidor da equipe ficará responsável pela submissão
%\item NÃO é permitido o uso de notebooks ou outro tipo de computador ou assistente pessoal;
\item Os problemas têm o mesmo valor na correção. 
\end{enumerate}
\item Sobre o arquivo de solução e submissão:
  \begin{enumerate}
  \item O arquivo de solução (o programa fonte) deve ter o mesmo nome que o especificado no enunciado (logo após o título do problema);
  \item confirme se você escolheu a linguagem correta e está com o nome de arquivo correto antes de submeter a sua solução;
  \item NÃO insira acentos no arquivo-fonte.
  \end{enumerate}
\item Sobre a entrada
  \begin{enumerate}
  \item A entrada de seu programa deve ser lida da entrada padrão (não use interface gráfica);
  \item Seu programa será testado em vários casos de teste válidos além daqueles apresentados nos exemplos. Considere que seu programa será executado uma vez para cada caso de teste.
  \end{enumerate}
\item Sobre a saída
  \begin{enumerate}
  \item A saída do seu programa deve ser escrita na saída padrão;
  \item Não exiba qualquer outra mensagem além do especificado no enunciado.
  \end{enumerate}
\end{enumerate}